Laravel is fairly easy to setup on a server, but it does require full control over your server's setup, which you generally won't have if you're using shared hosting.

There are various options:

\begin{itemize}
    \item You can use \href{https://devcenter.heroku.com/articles/getting-started-with-laravel}{Heroku}, which will host your site for free, but the server goes to sleep after 30 minutes of inactivity

    \item There's also \href{https://docs.aws.amazon.com/elasticbeanstalk/latest/dg/php-laravel-tutorial.html}{AWS}, which is often free for the first year of the most basic services. Just be warned, AWS can be \textit{deeply} confusing and if you forget to turn off your server before the free period runs out they'll start charging you.

    \item Another option would be to use a VPS host like Digital Ocean, but that does require a certain amount of \href{https://www.digitalocean.com/community/tutorials/how-to-deploy-a-laravel-application-with-nginx-on-ubuntu-16-04}{setup} and will cost \$5 a month. If you \href{https://m.do.co/c/41f18e2fa188}{sign up} you'll get \$10 free credit, which is enough to get a server running for a couple of months (\textbf{full transparency}: I get some free credit if you end up sticking with Digital Ocean after using that link)
\end{itemize}

You can use \href{https://forge.laravel.com/}{Laravel Forge} to make setting up all of the above servers easier. But it costs \$12 a month \textbf{on top} of the hosting fees, so it's probably not worth it unless you really don't want to deal with setting up servers.

\subsection*{Moral Support}

Setting up servers can be deeply scary early on (in fact, even after 10 years it's still anxiety-inducing at times). The only way to get better at it is to just keep doing it. It can take a long time, but it does all go in eventually.
