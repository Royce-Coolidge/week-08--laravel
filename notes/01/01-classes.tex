A class is an abstract representation of an object that you want to create. For example, you might have a class \texttt{Person} that allows you to create lots of object \textbf{instances} representing different people.
\\

Here's a class that represents a person:

\inputminted{php}{01/figures/01/01-class.php}

As you can see, it's much the same as a JavaScript class: we have the \texttt{class} keyword followed by the name of the class and there are some functions inside the class.
\\

As in JavaScript, we call functions that belong to an object \textbf{methods} and the values \textbf{properties}.

\begin{infobox}{Closing Tags}
    If you're been writing procedural PHP, you might be used to adding \texttt{?>} at the end of any PHP that you write. This allows you to go between HTML and PHP easily. However, when we're writing PHP with classes we'll never mix HTML and PHP.
    \\

    It's considered good practice to never use a PHP closing tag.
    \\

    From now on code examples won't include the opening tag, but you will need to add it as the first line in all your files.
\end{infobox}


However, there are some things we've not seen in JavaScript: the words \texttt{private} and \texttt{public}; and we declare our properties outside of the constructor method. We'll look at these in more detail shortly.
\\

Here's how we'd use our class:

\phpinputminted{01/figures/01/02-class-usage}

You can see that where we'd write a dot in JavaScript (\texttt{jim.getAge()}), we write an arrow in PHP (\texttt{\$jim->getAge()}), but otherwise it's almost identical in usage.

\begin{infobox}{PSR-2: Coding Style Guide}
    You've possibly noticed that in all the examples above the opening curly brace (\texttt{\{}) for classes and methods is on its own line. This is part of the \href{https://www.php-fig.org/psr/psr-2/}{PSR-2: Coding Style Guide} spec.
    \\

    If you do an \texttt{if} statement (or other control structure) then the opening curly brace, obviously, goes on \textit{the same} line.
    \\

    You're probably thinking that this doesn't make the slightest bit of sense. And you'd be right. PSR-2 was created by sending round a questionnaire about coding style to 30 or so of the most prolific PHP programmers and they just went with whatever the majority said for each point.
    \\

    But it's the style that everyone uses now. You'll get used to it.
\end{infobox}


\section{\texttt{\$this}}

Inside our classes we can use the \texttt{\$this} keyword to access properties and methods that belong to the current object instance. It works in much the same way as JavaScript except that it's much more reliable: \texttt{\$this} in PHP \textit{always} refers to the current object and has no meaning elsewhere.

\phpinputminted{01/figures/01/04-this}


\subsection{Returning \texttt{\$this}}

If your method doesn't have anything to return, for example if it just sets a value, then you can return \texttt{\$this}: it will give back the current object instance to the user, meaning that they can \textbf{chain} such methods together:

\phpinputminted{01/figures/01/05-returning-this}


\begin{infobox}{Using \texttt{new} Objects}
    Unlike in JavaScript you can't immediately use a created object:

    \begin{minted}{php}
        // won't work
        new Person("Jim", "Henson", "1936-09-24")->getAge();
    \end{minted}

    However, if you really want to, you can get around this with a pair of brackets around the \texttt{new} statement:

    \begin{minted}[frame=topline]{php}
        // will work
        // but you don't have a reference to the object anymore
        (new Person("Jim", "Henson", "1936-09-24"))->getAge();
    \end{minted}
\end{infobox}




\section{Properties}

In PHP we declare all of the properties that our class uses at the top of the class. This makes it easy to see which values are available. It also allows us to set default values easily:

\phpinputminted{01/figures/01/03-default-properties}


\section{Visibility}

Methods and properties in PHP can have three levels of visibility: \textbf{public}, \textbf{private}, and \textbf{protected}.
\\

A \textbf{public} method can be called anywhere in the PHP code. A \textbf{public} property can be read and changed from anywhere in the PHP code.
\\

A \textbf{private} method can only be called within the class it is declared in using \texttt{\$this}. A \textbf{private} property can only be read and changed within the class it belongs to by using \texttt{\$this}.

\phpinputminted{01/figures/01/06-visibility}

A \textbf{protected} property/method can only be used within the class it is declared in and any class that inherits from it (we'll look at inheritance later).
\\

In most instances it's good practice to make all of your properties private and then use ``getter'' and ``setter'' methods if you need to be able to change the values outside the class.


\section{Static Methods \& Properties}

Classes are primarily used for creating object instances. But sometimes it's useful to write some functionality about the object type instead of object instances.
\\

For example, if we have a \texttt{Person} class we might want to write a bit of functionality that gives us an alphabetic array of last names. We could write a \texttt{lastNames()} function, but then it's not associated with the \texttt{Person} class.
\\

Instead, we will write a \texttt{static} method: a method that belongs to the class itself rather than to an object instance.

\phpinputminted{01/figures/01/07-static}

Now it is clear that the \texttt{lastNames()} method has something to do with \texttt{Person} objects.
\\

Remember, \texttt{private} properties and methods can only be accessed using \texttt{\$this}, which doesn't have a value in \texttt{static} methods. So you'll need to use getters/setters just as you would if the function was written outside of the class.

\pagebreak

\begin{infobox}{Paamayim Nekudotayim}
    The \texttt{::} symbol is also known as the ``Paamayim Nekudotayim'', which is Hebrew for ``double colon''. This can lead to the somewhat mystifying error:

    \begin{minted}{diff}
        PHP expects T_PAAMAYIM_NEKUDOTAYIM
    \end{minted}

    All it's saying is you need a \texttt{::} somewhere.
    \\

    The \href{https://en.wikipedia.org/wiki/Zend_Engine}{Zend Engine}, which was behind PHP 3.0 and all subsequent releases, was originally developed at the Israel Institute of Technology.
\end{infobox}

Sometimes it's useful to be able to refer to the class you're currently working in. We can use the \texttt{static} keyword to do this:

\phpinputminted{01/figures/01/08-static-keyword}

\begin{infobox}{\texttt{static} vs \texttt{self}}
    You will sometimes see \texttt{self} instead of \texttt{static} to reference the current class. Using \texttt{static} in this way was only added in PHP 5.3, so a lot of older code uses \texttt{self}.
    \\

    If you're not using inheritance, then it doesn't make any difference which one you use. If you do then \texttt{self} refers to class that it is written in and \texttt{static} refers to the class it is called in (which might be different from where it was written if you're using inheritance).
\end{infobox}

You can also have \texttt{static} properties. Because they belong to the class they always exist, so you can use them for storing values that you want to have around. This is very useful for caching values.
\\

Say we needed to create a \texttt{\$renderer} object that all our object instances can use to render\textellipsis{} something. We could use a \texttt{static} property to store it, so that we only create it one time:

\phpinputminted{01/figures/01/09-static-property}

Privately declared \texttt{static} variables are accessible from inside object instances.



\section{Additional Resources}

\begin{itemize}[leftmargin=*]
    \item \href{https://phpapprentice.com/classes.html}{PHP Apprentice: Classes}
    \item \href{https://laracasts.com/series/php-for-beginners/episodes/12}{Laracasts: Classes 101}
    \item \href{https://laracasts.com/series/object-oriented-bootcamp-in-php/episodes/1}{Laracasts: Classes}
    \item \href{https://phpapprentice.com/static.html}{PHP Apprentice: Static}
\end{itemize}
