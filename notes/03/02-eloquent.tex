If we're going to create a database driven API, we'll need to interact with a database. There are two key parts to this:

\begin{itemize}
    \item Setting up and updating the structure of the database
    \item Reading/writing data from the database
\end{itemize}

Laravel makes it very easy to do both of these using the \texttt{artisan} command.
\\

\begin{infobox}{\texttt{artisan}}
    We'll be using \texttt{artisan} \textit{a lot} as it will save us writing all the boilerplate code.
    \\

    It's important to run all of your \texttt{artisan} commands inside the Vagrant box. Some of the commands work if you run them on your own machine, but not anything to do with databases. To avoid getting any errors \textit{always} run \texttt{artisan} inside Vagrant.

    \begin{minted}{bash}
        vagrant ssh # login box  - password is vagrant if needed
        cd code # Homestead puts everything inside the code directory
        artisan <artisan-command> # artisan command you wish to run
    \end{minted}

    It's probably easiest to have a terminal tab/window open that you just keep logged into the Vagrant box at all times.
    \\

    If you run just \texttt{artisan} it will list all of the commands it supports.
\end{infobox}



\section{Database Migrations}

The Homestead box creates a database called \texttt{homestead} which Laravel has been setup to use by the provisioning script.\footnote{Take a look in the \texttt{.env} file if you want to see.} The \texttt{homestead} database is empty by default, so we'll need to add the structure that we need for our API.
\\

Realistically, you're never going to know the complete structure of your database when you start building an API. Specifications will change over time and features will be added and removed. The structure you end up with at the end might be very different from what you started with.
\\

For this reason, we need a way to keep the structure of the database up-to-date with the rest of the app: it's no good writing code that tries to access a non-existent table or column.
\\

Keeping the database structure up-to-date becomes particularly tricky if there are multiple developers working on the same codebase. In the bad ole days developers would share a ``dump'' of the database: i.e. a complete copy of all of the tables, columns, and data. But this is bad for a couple of reasons:

\begin{itemize}
    \item Say you have two developers working on the API. One of them, Asha, is working on stuff to do with money and the various related tables. The other developer, Jiro, is working on stuff to do with users and the tables related to that. If Asha does a dump of her database and passes it over to Jiro, then it's not going to have all the user tables that Jiro has created, so he'd need to recreate them manually.

    \item Sharing the structure of the database is desirable, but we probably don't want to share the test data that other developers have been putting in. The things that people type in as test data are probably best kept to themselves. Anyone looking at a database I've populated manually would think I was obsessed with wombats, fish, and spoons.\footnote{And I really don't care for fish.}
\end{itemize}

This is where \textbf{database migrations} come in. A database migration is a file that contains instructions on how to update a database: for example, create a new table called \texttt{spoons} with \texttt{id INT}, \texttt{type VARCHAR(50)}, and \texttt{runcible TINYINT} columns. This file can then be run in order to update the structure of the database. A migration only needs to run once (the table only needs creating a single time), so there also needs to be a way to keep track of which database migration files have run and which haven't.


\subsection{Creating a Migration}

We're going to build a simple blog API which will allow us to create articles which have a title and an article body. So we want to create a table called \texttt{articles} with \texttt{id}, \texttt{title}, and \texttt{article} columns.
\\

Make sure you're inside the Vagrant box and in the \texttt{code} directory and then run:

\begin{minted}{shell}
    artisan make:model Article -m
\end{minted}

This creates an Article model class\footnote{We'll look at models shortly.} (\texttt{app/Article.php}) as well as a database migration (\texttt{database/migrations/<timestamp>\_create\_articles\_table.php}). It is possible to create database migrations \textit{without} creating a model (\texttt{artisan make:migration}), but we won't be needing that.
\\

Look inside the database migration and you'll see the following:

\inputminted{php}{03/figures/02/01-migration.php}

As you can see, a migration consists of an \texttt{up} method and a \texttt{down} method. The \texttt{up} method's job is to make the changes that we want to the database. The \texttt{down} method's job is to reverse the changes that \texttt{up} made: this allows us to ``rollback'' a migration if we made a mistake.
\\

You can see that both methods are already partly written for us. The \texttt{up} method is creating an \texttt{articles} table with an \texttt{id} column and the \texttt{down} method will remove the \texttt{articles} table. In fact, we won't need to change the \texttt{down} method at all.

\begin{infobox}{Timestamps}
    You'll notice that that \texttt{up} method also includes \texttt{\$table->timestamps()}. This adds \texttt{created\_at} and \texttt{updated\_at} columns, which Laravel will automatically keep up-to-date for us.
    \\

    These columns can be very useful when working with more complex data tables. Later in the week we'll look at a case when they are perhaps not necessary, but for now, it's best to keep them.
\end{infobox}

However, we do need to update the \texttt{up} method to include the other columns:

\stdinputminted[firstline=3]{php}{03/figures/02/02-migration-up.php}

We've added the \texttt{title} column as a ``string'' with maximum length of 100: in MySQL this is the same as \texttt{VARCHAR(100)}.\footnote{It doesn't use \texttt{varchar} because Laravel supports lots of different SQL engines, not all of which use the same types as MySQL.} We've also added a \texttt{article} column with the ``text'' type - this is for storing arbitrarily long bits of text.
\\

We've now created our database migration, but until we run it nothing will happen.\footnote{It's important to note that database migrations are not part of the Laravel app: they do not run whenever the app loads, only when you run them with \texttt{artisan}.} We run all migrations that haven't yet been run with:

\begin{minted}{shell}
    artisan migrate
\end{minted}

If you look at the database now you'll see that the \texttt{articles} table has been created.
\\

If you made a mistake, you can run \texttt{artisan migrate:rollback} to undo the last set of migrations that you ran, make any changes, and then run \texttt{artisan migrate} again.




\section{Models}

Next, we want to read and write data from the database. Back in the day we'd have done this by writing MySQL queries as strings in PHP and then using \texttt{mysqli} to run them.
\\

Having to write queries out as strings is pretty horrible at the best of times. But this would also limit our app to only being able to use MySQL, which would mean we couldn't easily switch it over to SQLite or PostgreSQL if we needed to.
\\

That's where \textbf{Object Relational Mapper}s (ORM) come in. An ORM lets you work with plain old PHP objects, but under the hood it's actually writing and running database queries for you. Laravel's ORM library is called \textbf{Eloquent}.\footnote{Good name.}

\pagebreak

\subsection{Writing Data}

We can use the \texttt{artisan tinker} command to run arbitrary bits of PHP code in our app environment. Let's create a new article:

\phpinputminted{03/figures/02/03-tinker-model}

Have a look in the database now and you'll see that a new row has been added with the data that we added to the \texttt{\$article} properties.
\\

The \texttt{Article} class was created when we ran \texttt{artisan make:model} earlier. If you take a look at the class it doesn't seem to do very much:

\phpinputminted{03/figures/02/04-Article}

All of the work here is being done by Eloquent's \texttt{Model} class, which \texttt{Article} is extending. The \texttt{Model} class does all of the databasey stuff for us. We just create a new \texttt{Article} class, set some properties, and then call the \texttt{save()} method. We don't need to worry about writing queries, setting up IDs, or updating the timestamps - Eloquent does it all for us.
\\

Once you've called the \texttt{save()} method, you can take a look at the \texttt{id} property:

\phpinputminted{03/figures/02/05-Article-id}

The \texttt{id} property now has a value because the article has been stored in the database. The \texttt{created\_at} and \texttt{updated\_at} properties also have values.
\\

If you make another change to the article and then save it again you'll see that the \texttt{updated\_at} property has been updated:

\phpinputminted{03/figures/02/06-Article-update}


\subsection{Reading Data}

We can also find articles using the \texttt{Article} class. It inherits various \texttt{static} methods from \texttt{Model}. For example, you can run \texttt{Article::all()} to get a \texttt{Collection} back with all the articles in it, and you can run \texttt{Article::find(1)} to find the article with the \texttt{id} 1.

\begin{minted}{php}
    // all the articles
    Article::all();

    // the article with id 1
    Article::find(1);
\end{minted}

You can also build up more complex \texttt{WHERE}-style queries, all just with Eloquent methods. See the \href{http://laravel.com/docs/master/eloquent}{Eloquent docs} for more information.



\section{Additional Resources}

\begin{itemize}[leftmargin=*]
    \item \href{http://laravel.com/docs/master/migrations}{Database Migrations}
    \item \href{http://laravel.com/docs/master/eloquent}{Eloquent}
\end{itemize}
